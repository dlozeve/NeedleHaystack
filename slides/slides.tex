\documentclass{beamer}

\usepackage{fontspec}

\setmainfont{Linux Libertine O}
\setsansfont{Linux Biolinum O}
%\setmonofont{Inconsolata}
\setmonofont[Scale=0.83]{Bitstream Vera Sans Mono}

\usepackage{polyglossia}
\setdefaultlanguage{english}

\usepackage{graphicx}
\usepackage{xcolor}
\usepackage{wrapfig}
\usepackage{subfig}
\usepackage{lettrine}
\usepackage{booktabs}
\usepackage{multirow}
\usepackage{amsmath,amssymb}

\usepackage{smartdiagram}

\usepackage{pdfpages}

\usepackage{microtype}

%% Propriétés du document PDF
%\usepackage[hidelinks]{hyperref}

% \hypersetup{
%   pdfauthor={Dimitri Lozeve},
%   pdftitle={Soutenance de stage DFHM},
%   pdfsubject={},
%   pdfkeywords={stage,soutenance,polytechnique,3rg}
% }

\usepackage{minted}
\setminted[java]{linenos,autogobble,fontsize=\small}



\author{Dimitri \textsc{Lozeve} \and Antoine \textsc{Ogier}}
\institute{École polytechnique --- Promotion X2014}
\date{January 22, 2016}
\title{A Needle in a Haystack}
\subtitle{Programming Project}







%%% Local Variables:
%%% mode: latex
%%% TeX-master: "slides"
%%% End:


\usetheme{Pittsburgh}
\setbeamertemplate{footline}[frame number]

% \usebackgroundtemplate%
% {%
%     \includegraphics[width=\paperwidth,height=\paperheight]{images/background.jpg}%
% }

\begin{document}


\begin{frame}
  \titlepage\
\end{frame}

\begin{frame}
  \frametitle{Introduction}

  \begin{block}{String matching algorithms}
    \begin{itemize}
    \item large amounts of \emph{text} (data science, etc.): need of
      efficient algorithms to get meaningful \emph{information} out of
      huge amounts of raw \emph{data}
    \item rise of genomics (human genome sequenced in 2003): finding
      genes, sequences, edit distances $\rightarrow$ direct
      applications
    \item interesting for the algorithms themselves
    \end{itemize}
  \end{block}
\end{frame}

\begin{frame}
  \frametitle{Plan}

  \begin{enumerate}
  \item Presentation of the algorithms, implementation, issues
    encountered, and tests
    \begin{itemize}
    \item Naive algorithm
    \item Karp-Rabin
    \item Knuth-Morris-Pratt
    \item Boyer-Moore
    \end{itemize}
  \item Comparison and discussion
  \end{enumerate}
\end{frame}

\section{Naive algorithm}
\label{sec:naive-algorithm}

\begin{frame}
  \frametitle{Naive algorithm}
  
  \begin{block}{The most straightforward}
    \begin{itemize}
    \item easy to understand
    \item easy to implement
    \item terrible performance: $\mathcal{O}(nm)$
    \end{itemize}
  \end{block}

  \begin{block}{Implementation details}
    Use of the \mintinline{java}|String.substring| method: same
    complexity.
  \end{block}
  
\end{frame}

\begin{frame}
  \frametitle{Naive algorithm: tests}
  
  \begin{table}[ht]
    \centering
    \tiny
  \begin{tabular}{lllll}
    \toprule[1pt]
    & \multicolumn{2}{c}{\emph{F.\ succinogenes}} & \multicolumn{2}{c}{Tolstoy} \\
    Pattern length & Time & Time/length & Time & Time/length \\
    \cmidrule(r){1-1} \cmidrule(lr){2-3} \cmidrule(lr){4-5}
    10 & 107 & 10.7 & 80 & 8 \\
    20 & 100 & 5 & 78 & 3.9 \\
    30 & 108 & 3.6 & 79 & 2.63 \\
    40 & 102 & 2.55 & 90 & 2.25 \\
    50 & 105 & 2.1 & 84 & 1.68 \\
    60 & 99 & 1.65 & 105 & 1.75 \\
    70 & 124 & 1.77 & 106 & 1.51 \\
    80 & 131 & 1.64 & 119 & 1.49 \\
    90 & 129 & 1.43 & 149 & 1.66 \\
    100 & 122 & 1.22 & 176 & 1.76 \\
    150 & 175 & 1.17 & 205 & 1.37 \\
    200 & 397 & 1.99 & 299 & 1.50 \\
    300 & 314 & 1.05 & 430 & 1.43 \\
    400 & 624 & 1.56 & 575 & 1.44 \\
    500 & 507 & 1.01 & 620 & 1.24 \\
    1000 & 932 & 0.93 & 1008 & 1.008 \\
    2000 & 1796 & 0.89 & 1748 & 0.87 \\
    \bottomrule[1pt]
  \end{tabular}
  \caption{\small Naive algorithm tested on various input texts. Times are in milliseconds. Text lengths = 3 842 635 and 3 226 641.}
\end{table}
\end{frame}

\section{Karp-Rabin}
\label{sec:karp-rabin}

\begin{frame}
  \frametitle{Karp-Rabin}

  \begin{block}{A semi-numerical algorithm}
    \begin{itemize}
    \item use of a \emph{hash function:}
      \[ h(q) = \sum_{i=0}^{m-1} q_i 2^{m-i-1} \bmod w \]
    \item based on \emph{arithmetic} instead of character comparisons
    \item linear time without any significant preprocessing (hash
      function only called twice)
    \item trade-off: false matches may arise (hash function collisions)
    \end{itemize}
  \end{block}

\end{frame}

\begin{frame}
  \frametitle{Karp-Rabin: implementation and pitfalls}

  \begin{itemize}
  \item Update the hash without computing too many powers of 2
  \item Bounded \mintinline{java}|int| type: use of
    \mintinline{java}|java.math.BigInteger| class?
    \begin{itemize}
    \item[$\rightarrow$] big potential increase of running time (and
      memory used)
    \end{itemize}
  \item different hash function (fewer collisions)?
    \begin{itemize}
    \item[$\rightarrow$] must maintain the possibility of
      \emph{updating} the hash, not recomputing it each time we shift
    \end{itemize}
  \end{itemize}
\end{frame}

\begin{frame}
  \frametitle{Karp-Rabin: tests}
  
  \begin{block}{Search of \texttt{"ATATCTAACT"} in \emph{A.\ fumigatus} genome}
    \begin{itemize}
    \item Naive algorithm: 594~ms
    \item Karp-Rabin: 269~ms
    \end{itemize}
  \end{block}

  \begin{block}{Search of \texttt{"trying"} in Tolstoy's \emph{War and Peace}}
    \begin{itemize}
    \item Naive algorithm: 62~ms
    \item Karp-Rabin: 38~ms
    \end{itemize}
  \end{block}
\end{frame}


\section{Knuth-Morris-Pratt}
\label{sec:knuth-morris-pratt}

\begin{frame}
  \frametitle{Knuth-Morris-Pratt}

  \begin{block}{New approach}
    \begin{itemize}
    \item best-known linear-time algorithm
    \item analyze the repetition of prefixes
    \item complexity: preprocessing in $\mathcal{O}(m)$ and search in
      $\mathcal{O}(n)$
    \item complexity easily proven
    \item same algorithm for preprocessing and search, easily
      explained and understood
    \end{itemize}
  \end{block}
\end{frame}

\begin{frame}
  \frametitle{Knuth-Morris-Pratt: tests}

  \begin{table}[ht]
    \centering
    \begin{tabular}{lllllllll}
      \toprule[1pt]
      Pattern length & 10 & 20 & 50 & 100 & 200 & 500 & 1000 & 2000 \\
      \midrule
      Tolstoy & 33 & 28 & 27 & 28 & 28 & 29 & 27 & 28 \\
      \emph{F.\ succinogenes} & 46 & 47 & 42 & 43 & 41 & 46 & 41 & 43 \\
      \emph{A.\ fumigatus} & 250 & 225 & 210 & 229 & 219 & 215 & 234 & \\
      \bottomrule[1pt]
    \end{tabular}
    \caption{Knuth-Morris-Pratt running time on various texts, with varying pattern length. Times in milliseconds. Text lengths = 3~226~641, 3~842~635 and 29~385~546.}
  \end{table}
\end{frame}


\section{Boyer-Moore}
\label{sec:boyer-moore}

\begin{frame}
  \frametitle{Boyer-Moore}

  \begin{block}{Clever ideas}
    \begin{itemize}
    \item right-to-left scan
    \item (extended) bad character shift rule: better for large
      alphabets (e.g.\ natural languages)
    \item (strong) good suffix shift rule: necessary for linear time,
      better for short alphabets (e.g. DNA)
    \item complexity: preprocessing in $\mathcal{O}(m)$, search in
      $\mathcal{O}(n)$
    \item complexity difficult to prove
    \end{itemize}
  \end{block}
\end{frame}

\begin{frame}
  \frametitle{Boyer-Moore: issues}

  \begin{block}{Difficult implementation}
    \begin{itemize}
    \item hard to understand (not natural)
    \item long and difficult to implement (a lot of different cases)
    \item necessity to split into different methods
      (\mintinline{java}|badCharacterRule|,
      \mintinline{java}|goodSuffixRule|, \mintinline{java}|shift|,
      \mintinline{java}|boyermoore|)
    \item computing the shift: easy to get lost in indexes
    \item various data structures (array, \mintinline{java}|HashMap|,
      \mintinline{java}|ArrayList| vs.\ \mintinline{java}|LinkedList|)
    \end{itemize}
  \end{block}
\end{frame}

\begin{frame}
  \frametitle{Boyer-Moore: tests}

  \begin{table}[ht]
    \centering
    \begin{tabular}{lllllllll}
      \toprule[1pt]
      Pattern length & 10 & 20 & 50 & 100 & 200 & 500 & 1000 & 2000 \\
      \midrule
      Tolstoy & 38 & 33 & 29 & 48 & 31 & 32 & 34 & 32 \\
      \emph{F.\ succinogenes} & 70 & 50 & 33 & 29 & 29 & 23 & 28 & 42 \\
      \emph{A.\ fumigatus} & 209 & 147 & 215 & 85 & 72 & 106 & 114 & \\
      \bottomrule[1pt]
    \end{tabular}
    \caption{Boyer-Moore running time on various texts, with varying pattern length. Times in milliseconds. Text lengths = 3~226~641, 3~842~635 and 29~385~546.}
  \end{table}
\end{frame}


\section{Comparisons}
\label{sec:comparisons}


\begin{frame}
  \frametitle{Comparisons}

  \begin{table}[ht]
    \centering
    \begin{tabular}{lcccccc}
    \toprule[1.5pt]
    & \multicolumn{2}{c}{Naive} & \multicolumn{2}{c}{KMP} & \multicolumn {2}{c}{Boyer-Moore} \\
    & short & long & short & long & short & long \\
    \cmidrule(r){1-1} \cmidrule(lr){2-3} \cmidrule(lr){4-5} \cmidrule(l){6-7}
    Verne & 36 & 59 & 17 & 10 & 29 & 15 \\
    Tolstoy & 77 & 131 & 15 & 11 & 47 & 6 \\
    \emph{F.\ succinogenes} & 364 & 5329 & 28 & 28 & 13 & 11 \\
    \emph{A.\ fumigatus} & 662 & 43739 & 205 & 214 & 139 & 85 \\
    \bottomrule[1.5pt]
  \end{tabular}
  \caption{Comparisons of the most common algorithms applied to short and long patterns. Times in milliseconds.}
\end{table}
\end{frame}


\begin{frame}
  \frametitle{Comparisons}

  \begin{block}{Boyer-Moore}
    \begin{itemize}
    \item better with long patterns
    \item theoretically worse than KMP for small alphabets: not for
      our implementation!
    \item real improvement when increasing the pattern length; not so
      much when increasing text length
    \end{itemize}
  \end{block}

  \begin{block}{Knuth-Morris-Pratt}
    \begin{itemize}
    \item much easier to understand and implement
    \item excellent overall performance
    \item do not perform well on DNA
    \item possibility to turn into a \emph{real-time} string matching
      algorithm
    \end{itemize}
  \end{block}
\end{frame}

\begin{frame}
  \frametitle{Conclusion}

  \begin{itemize}
  \item Best general-case algorithm: Knuth-Morris-Pratt
    \begin{itemize}
    \item[$\rightarrow$] easier to implement
    \item[$\rightarrow$] good overall performance
    \end{itemize}
  \item Important to understand Boyer-Moore, and when it could perform
    better
  \item Semi-numerical algorithms: excellent concept, useful in
    specific situations (e.g.\ limited memory)
  \item \emph{Suffix trees:} preprocessing in $\mathcal{O}(n)$, search
    in $\mathcal{O}(m)$
    \begin{itemize}
    \item[$\rightarrow$] huge advantage when multiple searches on the
      same text are needed (genomics).
    \end{itemize}
  \end{itemize}
\end{frame}


\end{document}


%%% Local Variables:
%%% mode: latex
%%% TeX-master: t
%%% End:
