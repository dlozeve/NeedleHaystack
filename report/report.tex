\documentclass[a4paper,11pt,openany,extrafontsizes,twoside,article]{memoir}

\usepackage{fontspec}

\setmainfont{Linux Libertine O}
\setsansfont{Linux Biolinum O}
%\setmonofont{Inconsolata}
\setmonofont[Scale=0.83]{Bitstream Vera Sans Mono}

\usepackage{polyglossia}
\setdefaultlanguage{english}

\usepackage{graphicx}
\usepackage{xcolor}
\usepackage{wrapfig}
\usepackage{subfig}
\usepackage{lettrine}
\usepackage{booktabs}
\usepackage{multirow}
\usepackage{amsmath,amssymb}

\usepackage{smartdiagram}

\usepackage{pdfpages}

\usepackage{microtype}

%% Propriétés du document PDF
%\usepackage[hidelinks]{hyperref}

% \hypersetup{
%   pdfauthor={Dimitri Lozeve},
%   pdftitle={Soutenance de stage DFHM},
%   pdfsubject={},
%   pdfkeywords={stage,soutenance,polytechnique,3rg}
% }

\usepackage{minted}
\setminted[java]{linenos,autogobble,fontsize=\small}



\author{Dimitri \textsc{Lozeve} \and Antoine \textsc{Ogier}}
\institute{École polytechnique --- Promotion X2014}
\date{January 22, 2016}
\title{A Needle in a Haystack}
\subtitle{Programming Project}







%%% Local Variables:
%%% mode: latex
%%% TeX-master: "slides"
%%% End:


\begin{document}

\pagestyle{Ruled}
\tightlists{}

\maketitle

%% Style de chapitre
%\chapterstyle{hangnum}

\begin{abstract}
  In this report, we will present four algorithms for the exact string
  matching problem: the naive algorithm, and the algorithms from
  Karp-Rabin~\cite{karp1987efficient},
  Knuth-Morris-Pratt~\cite{knuth1977fast}, and
  Boyer-Moore~\cite{boyer1977fast}. We also present a standard
  implementation of these algorithms in Java, and some complexity and
  peformance analysis.
\end{abstract}

\tableofcontents*

\chapter{Problem statement}
\label{cha:problem-statement}

In all algorithm descriptions and implementation, we will consider a
string $S$ of length $n$, which represents the text to search, and a
pattern $q$ of length $m$. The algorithms will have to find all the
\emph{exact matches} of the pattern in the text, and return a list of
positions at which the pattern occurs in the text.

We chose to implement the various algorithms in Java. The testing has
been done with large texts, such as the novels \emph{Mysterious
  Island} by Jules Verne or \emph{War and Peace} by Tolstoy, but also
with the (large) genomes of \emph{Fibrobacter succinogenes} and
\emph{Aspergillus fumigatus}. That way, we were able to test our
implementations against text and patterns of various absolute and
relative size, and with various alphabet sizes (we will see later how
this can be important for performance analysis).

Our code has been written using the Git version control system. Its
repository is hosted on Bitbucket at
\url{https://bitbucket.org/dlozeve/needlehaystack}.

\chapter{Testing interface}
\label{cha:testing-interface}

In order to easily test our algorithms, we created a class
\mintinline{java}|Main| which allowed us to test and compare the
performance of various algorithms using simple command-line arguments.

After reading the text file and the pattern (given as a simple string
argument), we parse the command-line options, activating the
algorithms chosen by the user. A simple test function with timing for
an algorithm looks like this (here for Knuth-Morris-Pratt):

\begin{minted}{java}
if (kmp) {
    long startTime = System.nanoTime();
    LinkedList<Integer> res = KnuthMorrisPratt.kmp(text,query);
    long endTime = System.nanoTime();
    System.out.println("********** Knuth-Morris-Pratt **********\n"
        + "Result: " + res.toString() + "\n"
        + "Time: " + (endTime - startTime)/1000000 + " ms\n"
        + "****************************************\n\n");
}
\end{minted}

The main function thus gives us an output like this:

\begin{minted}{text}
********** Knuth-Morris-Pratt **********
Result: [29218957]
Time: 213 ms
****************************************
\end{minted}

\chapter{Naive algorithm}
\label{cha:naive-algorithm}

\section{Description}
\label{sec:description}

The naive algorithm consists simply in trying all the positions for
the pattern in the text, and at each position, test the equality of
each character in the pattern with the corresponding character in the
text.

\section{Implementation}
\label{sec:implementation}

\begin{minted}{java}
    public static LinkedList<Integer> naive(String s, String q) {
        LinkedList<Integer> res = new LinkedList<>();

        int n = s.length();
        int m = q.length();		

        for (int i = 0; i < n-m; i++) {
            if (s.substring(i, i+m).equals(q)) {
                res.add(i);
            }
        }
        return res;		
    }
\end{minted}

Here we used the method \mintinline{java}|String.substring(int, int)|
in order to get a more simple and concise version of the
algorithm. However, it should be mentioned that this is internally
equivalent to a loop testing each character one after the other. Thus,
this does not affect the complexity of the algorithm.

In order to retrieve and return all the positions, we use a
doubly-linked list, as implemented in the standard library class
\mintinline{java}|java.util.LinkedList|\footnote{\url{https://docs.oracle.com/javase/8/docs/api/java/util/LinkedList.html}}. This
allows us to add to our result the occurrences found, without being
limited in the number of such occurrences. This approach will be used
throughout all the algorithms subsequently implemented.

\section{Test and performance analysis}
\label{sec:test-perf-analys}

The worst-case complexity of the naive algorithm is obviously
$\mathcal{O}(nm)$. Even a slight increase in the pattern length can lead to
unpractical running times. This algorithm is particularly not adapted
for genome searches, where one often needs to search extremely long
texts with long patterns (e.g.\ to locate a particular gene).

For instance, for a search of a 1474-character pattern in
\emph{aspergillus fumigatus} genome, the naive algorithm runs in more
than 10 seconds, on a recent computer.


\chapter{Karp-Rabin algorithm}
\label{cha:karp-rabin-algorithm}

\section{Description}
\label{sec:description-1}

The Karp-Rabin algorithm is described as a \emph{semi-numerical}
algorithm, because it uses hash codes in order to speed up
comparisons. The hash code is a very simple in this case; it is
defined by:
\[ h(q) = \sum_{i=0}^{m-1} q_i 2^{m-i-1} \bmod w, \] where $q_i$ is
the numerical value of the $i$-th character in $q$, and w is a (large)
prime.

This hash function is very convenient because we can compute easily
and cheaply the hash for $s[i+1..i+m]$ knowing the hash for
$s[i..i+m-1]$, by simple constant-time operations (modular
substractions, additions and binary shifts).

When looping through the text, we only compare the hashes. There is
only need to check for an occurrence character by character if the
hashes match, otherwise we are sure that the substrings are different.

\section{Implementation and limitations}
\label{sec:impl-limit}

We compute the hash function in linear-time:

\begin{minted}{java}
    public static int hash (String s, int w) {
        int hash = 0;
        int m = s.length();
        int power = 1;

        for (int i = m-1; i >= 0; i--) {
            hash = (hash +  s.charAt(i)*power) % w;
            power = power*2;
        }
        return hash;
    }
\end{minted}

Note that this function is only needed twice: for the pattern and for
the first part of the text. After that, we only need to update the
current hash to follow the text.

The rest of the implementation is pretty straightforward, and not very
different from the naive algorithm. Here is the central part, where
the hash is updated:

\begin{minted}{java}
        // We'll need this in the loop, so we only compute it once.
        int pow = (int) Math.pow(2, m-1);

        for (int i = 0; i < n-m; i++) {
            if (hash == queryHash && s.substring(i, i+m).equals(q)) {
                res.add(i);
            }
            // Update the hash
            hash = ((hash - s.charAt(i)*pow)*2 + s.charAt(i+m)) % w;
        }
\end{minted}

This algorithm leads to a significant improvement over the naive
algorithm. However, it suffers from a huge implementation problem: the
\mintinline{java}|int| type is bounded. Because of this issue, present
in most languages, we cannot search for patterns too long (namely,
longer than 32 characters), because the value $2^{m-1}$ used to update
the hash would be greater than the maximum \mintinline{java}|int|
value.

In order to solve this issue, we might replace the
\mintinline{java}|int| type with the unbounded
\mintinline{java}|BigInteger|
class\footnote{\url{https://docs.oracle.com/javase/8/docs/api/java/math/BigInteger.html}},
part of the \mintinline{java}|java.math| library. The implementation
becomes more difficult, and less readable (use of methods like
\mintinline{java}|BigInteger.multiply(BigInteger, BigInteger)| instead
of simply \mintinline{java}|*|), but it allows the algorithm to work
for any pattern length. However, it is not clear whether the trade-off
in complexity (\mintinline{java}|BigInteger| is considerably slower
than normal \mintinline{java}|int|) is worth the trouble. It is our
opinion that it is easier \emph{and} better to use one of the
following algorithms instead.

\chapter{Knuth-Morris-Pratt algorithm}
\label{cha:knuth-morris-pratt}

\section{Description}
\label{sec:description-3}

The Knuth-Morris-Pratt algorithm preprocesses the pattern to be
searched for before performing the actual search. More precisely, for
each position $j$, the alorithm stores the index of the largest prefix
of $q$ that is also a suffix of $q[0..j-1]$, called
$\mathtt{next}(j)$. It then uses this information to check each
character in the text only once.

\subsection{Preprocessing}
\label{sec:preprocessing}

The principle of the preprocessing derives from a simple observation:
if the prefix $q[0..\mathtt{next}(k)]$ is a suffix of $q[0..k]$, then
$q[0..\mathtt{next}(k)-1]$ is a suffix of $q[0..k-1]$. As such, if
$\mathtt{next}(i)$ is known for $i$ from $0$ to $k-1$, we can compute
$\mathtt{next}(k)$ as follows: if $q[0..\mathtt{next}(k-1)-1]$ is a
suffix for $q[0..k]$, it is necessarily the longest one (otherwise
there would be a longer prefix of $q$ that is also a suffix of
$q[0..k-1]$, which is impossible by definition of
$\mathtt{next}(k-1)$). If it is not, the next possibility is a prefix
of $q[0..\mathtt{next}(k-1)-1]$ that is also a suffix of $q[0..k-1]$
and thus of $q[0..\mathtt{next}(k-1)-1]$ for length reasons. For this
reason we retrieve $\mathtt{next}(\mathtt{next}(k-1)-1)$ and proceed
to the same checks, and so on until we find a fitting prefix or
exhaust all possibilities.

\subsection{Search}
\label{sec:search}

Once the preprocessing is complete, the search is quite
straightforward. Suppose we are checking $S$ at position $i$ to
compare it with position $k$ in q. Two possibilities can arise:
\begin{itemize}
\item The characters match. In that case we can advance to the next
  position in both the pattern and the text. If the end of the pattern
  has been reached, we have found a new occurence.
\item There is a mismatch. In that case, if there is an occurence of
  $q$ in $S$ beginning at position $i-k < j < i$, the substring
  between $i-1$ and $j$ is both a prefix of $q$ and a suffix of
  $q[0..k-1]$. Consequently, we resume our check starting at position
  $\mathtt{next}(k-1)$ in $q$.
\end{itemize}

\section{Implementation}
\label{sec:implementation-2}

\begin{minted}{java}
    public static int[] preprocess (String q) {
        int m = q.length();
        // next[k] stores the length of the maximal substring of q
        // ending at k that matches a prefix of q.
        int[] next = new int[m];
        // l is the length we want to compute.
        int l = 0;

        // We go through the string with k.
        for (int k=1; k<m; k++) {
            // At each point, we compare the next character in q
            // with the next character after the prefix already computed.
            // If they do not match, we check again for a smaller prefix,
            // (that is identical to a suffix of the previous prefix).
            // If the previous prefix was q[0..l-1],
            // the new prefix will be q[0..next[l-1]].
            while (l > 0 && q.charAt(l) != q.charAt(k)) {
                l = next[l-1];
            }
            // If the characters match, we can increment the prefix length.
            if (q.charAt(l) == q.charAt(k)) {
                l++;
            }
            // We can now store the computed prefix length in the array:
            next[k] = l;
        }
        return next;
    }

    public static LinkedList<Integer> kmp (String s, String q) {
        LinkedList<Integer> res = new LinkedList<>();
        int n = s.length();
        int m = q.length();
        int[] next = preprocess(q);

        // This is nearly the same algorithm as in preprocessing.
        // But this time, we have *two* strings to compare.

        // k represents the length of the maximal prefix of q
        // that matches a substring of s beginning at i.
        int k = 0;

        // i is the position in s.
        for (int i =0; i<n; i++) {
            // If the next characters do not match,
            // we shift q of the maximal amount possible,
            // using the array next computed during prepocessing.
            while (k > 0 && q.charAt(k) != s.charAt(i)) {
                k = next[k-1];
            }
            // If the next characters do match,
            // we go to the next characters (of both strings).
            if (q.charAt(k) == s.charAt(i)) {
                k++;
            }
            // When k==m, we have reached the end of q,
            // and so we have found a match!
            if (k == m) {
                res.add(i-k+1);
                // We can go now to the next step to find the next match.
                k = next[k-1];
            }
        }
        return res;
    }
\end{minted}

\section{Performance analysis}
\label{sec:performance-analysis}

\subsection{Preprocessing}
\label{sec:preprocessing-1}

At any step in the algorithm, two situations can arise. If the
operation is performed as a part of the \mintinline{java}|while| loop,
$l$ decreases while $k$ is unchanged, thus $k-l$ increases. When the
\mintinline{java}|while| loop ends, $k$ increases by one and $l$
increases by at most one. As such, $k-l$ does not decrease. We see
that in any case, both $k$ and $k-l$ are nondecreasing and at least
one of these quantities increases strictly. Since both are bounded by
$m$ the length of $q$, this shows that the preprocessing is in
complexity $\mathcal{O}(m)$.

\subsection{Search}
\label{sec:search-1}

The arguments for the complexity of the search are mostly the same as
those for the preprocessing. At each step, either the position $i$ or
$i-k$ (the starting position of the occurence being checked)
increases, leading to complexity $\mathcal{O}(n)$.


\chapter{Boyer-Moore algorithm}
\label{cha:boyer-moore-algor}

\section{Description}
\label{sec:description-2}

The ideas behind the Boyer-Moore algorithms derive from the fact that
even if Knuth-Morris-Pratt has a worst-case linear time complexity, we
still go through every character in the text, either to compare it to
the pattern or to determine that we don't need to compare
it. Boyer-Moore uses a \emph{right-to-left} scan, which allows
skipping some parts.

Moreover, it combines two different rules to determine the shift at
each step, the \emph{bad character rule} and the \emph{good suffix
  rule}.

\subsection{Bad character rule}
\label{sec:bad-character-rule}

When scanning from right to left the pattern against the text, we
encounter two different characters. The idea is to shift the pattern
$q$ (to the right) in order to match the rightmost occurrence of the
last matched character with the corresponding character in $S$.

For this, we need to build a table of the rightmost position in $q$ of
all the letters in the alphabet. However, this simple version of the
bad character rule does not allow for shifting when we are already on
the left of the rightmost character.

That is why we chose to implement the \emph{extended} bad character
rule: each cell in the array will contain not only the rightmost
position of the character, but a list of the position of the character
in $q$, sorted from right to left.

Thus, when we reach a mismatch, we can look for the rightmost
occurrence of the last matched character, on the left of the current
position, simply by going through the list.

The preprocessing for this rule is easy and can be realised in linear
time, that is $\mathcal{O}(m)$. It is indeed enough to go once through the
pattern, appending the current position to the list corresponding to
the current character.

\subsection{Good suffix rule}
\label{sec:good-suffix-rule}

We can easily observe that the bad character rule is not enough by
itself to achieve good performance (it is not even enough to prove
linear-time complexity). It is particularly limited when the alphabet
is small (in genomes for instance, when the alphabet is reduced to
letters A, T, C and G).

Therefore, the good suffix rule can achieve much better results, when
allied to the bad character rule. It is defined like this:
\begin{itemize}
\item If a substring of $S$ matches a suffix $t$ of $q$, but a mismatch
  occurs at the next comparison on the left:
  \begin{itemize}
  \item If it exists, find the rightmost copy $t'$ of $t$ in $q$, such
    that the character left of $t'$ is different from the character
    left of $t$, and shift so that $t'$ matches the substring of $S$;
  \item If such a copy does not exist, find the longest suffix of $t$
    that is also a prefix of $q$ (this is called a \emph{border}), and
    shift $q$ accordingly;
  \item Otherwise, shift $q$ by $m$ places.
  \end{itemize}
\item If an occurrence of $q$ has been found in $S$:
  \begin{itemize}
  \item Find the largest proper border of $q$, and shift $q$ so that
    the prefix match the suffix in $S$;
  \item If such a border does not exist, shift $q$ by m places.
  \end{itemize}
\end{itemize}

The preprocessing for this rule is much more complicated, but can
nevertheless be done in $\mathcal{O}(m)$ time. We shall see the details of the
algorithm while reviewing the implementation.

\subsection{Combining the rules}
\label{sec:combining-rules}

By shifting according to one of these rules, we are guaranteed that we
will not miss any occurrence of $q$ in $S$. Therefore, we can simply
compute the shift determined by the two rules, and shift $q$ at each
step by the maximum allowed.

The main Boyer-Moore algorithm, after the preprocessing stage, simply
compares right-to-left $q$ with $S$, and when a mismatch occurs or
when the beginning of $q$ is reached, computes the shift according to
the two rules.

\section{Implementation}
\label{sec:implementation-1}

\subsection{Bad character rule preprocessing}
\label{sec:bad-character-rule-1}

\begin{minted}{java}
    /* This method implements the "bad character rule":
     * it computes an array (or here, a HashMap) of positions of each character
     * of the alphabet.
     */
    public static Map<Character, ArrayList<Integer>> badCharacterRule (String q) {
        int m = q.length();

        // We use a HashMap that associates to each character a list of its positions
        // in the String q.
        Map<Character, ArrayList<Integer>> r = new HashMap<Character, ArrayList<Integer>>();

        for (int i=m-1; i>=0; i--) {
            // For each character in q, if it is the first time we encounter it,
            // we create a List to hold its position.
            if (r.get(q.charAt(i)) == null) {
                r.put(q.charAt(i), new ArrayList<Integer>());
            }
            r.get(q.charAt(i)).add(i);
        }
        return r;
    }
\end{minted}

In this implementation, we do not compute an array: we do not know
what the length of the alphabet will be, and we wanted to extend the
search to the largest charset possible. (This implementation should
work fine with any UTF-8 character.) Moreover, it is more
space-efficient not to declare an array the size of the alphabet if
only a few characters are actually used in the pattern.

Therefore, we implemented the rule using a
\mintinline{java}|HashMap<Character, ArrayList<Integer>>| instead of
an array, which allows us to append cheaply and easily its position to
the list corresponding to a character.

We use \mintinline{java}|ArrayList| instead of
\mintinline{java}|LinkedList| because we will later need to go through
the list (cf.~Section~\ref{sec:computing-shift}), and it is more
convenient to access directly the $j$-th element rather than popping
and pushing things around. (Accessing an element of an
\mintinline{java}|ArrayList| runs in amortized constant time, whereas
with \mintinline{java}|LinkedList| it runs in linear
time\footnote{\url{https://docs.oracle.com/javase/8/docs/api/java/util/ArrayList.html}}.)

\subsection{Good suffix rule preprocessing}
\label{sec:good-suffix-rule-1}

To implement the good suffix rule, we need to compute two arrays. The
first one, \mintinline{java}|f|, stores in its $i$-th cell the
beginning of the rightmost copyof the suffix beginning at $i$ (case
1.1). The second one, \mintinline{java}|l|, stores the \emph{borders}
of $q$, that is, \mintinline{java}|l[i]| stores the beginning of the
largest suffix of $q[i..m-1]$ which is also a prefix of $q$. It is
needed in cases 1.2 and 2.1 (when there was an occurrence of $q$ in
$S$).

To compute them, we use a list called \mintinline{java}|suffixCopies|,
which holds the indexes of the copies of the current suffix.

Then we loop through $q$ from right to left. We begin by testing,
using the last (leftmost) element of \mintinline{java}|suffixCopies|,
whether the current suffix is a prefix, and if that is the case, we
complete \mintinline{java}|l[i]| accordingly:

\begin{minted}{java}
            // if the leftmost copy is a prefix, we add it to the array l.
            if (suffixCopies.size() > 0 && suffixCopies.get(suffixCopies.size()-1) == 0) {
                l[i] = i;
            }
            else { // otherwise the previous suffix is still the best border available:
                l[i] = l[i+1];
            }
\end{minted}

Then comes the update of \mintinline{java}|suffixCopies| in order to
complete \mintinline{java}|f|.

For each copy in \mintinline{java}|suffixCopies|, if the character on
its left is equal to the character on the left of the suffix, it
remains a copy. Otherwise, it is not a copy anymore, so we remove it,
but we can fill \mintinline{java}|f[i]| with its position:

\begin{minted}{java}
            // We go through suffixCopies, starting at the end
            for (int j=suffixCopies.size()-1; j>=0; j--) {
                // We retrieve the index of the copy of the current suffix
                int k = suffixCopies.get(j);
                // If k==0, this copy is useless now, we can remove it.
                if (k == 0) {
                    suffixCopies.remove(j);
                }
                // If the next characters (on the left) are not the same,
                else if (q.charAt(k-1) != q.charAt(i-1)) {
                    // we put the link to the copy in the result array
                    f[i] = k;
                    // and we delete this index, since it isn't a correct copy anymore.
                    suffixCopies.remove(j);
                }
                else if (q.charAt(k-1) == q.charAt(i-1)) {
                    // If the characters on the left are the same,
                    // we add the next character to the copy.
                    suffixCopies.set(j, k-1);
                }
            }
\end{minted}

At the end, we need to return both \mintinline{java}|f| and
\mintinline{java}|l|. To avoid declaring them as global variables, we
return an
\mintinline{java}|Object[]{f,l}|\footnote{\url{http://stackoverflow.com/questions/6337075/return-two-arrays-in-a-method-in-java}}.

\subsection{Computing the shift}
\label{sec:computing-shift}

We define a method \mint{java}|int shift(String q, Map<Character,
ArrayList<Integer>> r, int[] f, int[] l, int i)| that computes the
shift according to the two rules.

First, we go through the list of positions of the current character to
find its nearest sibling on its left:

\begin{minted}{java}
        // Bad character rule
        // This list contains all the positions of the character at i in q.
        ArrayList<Integer> positions = r.get(q.charAt(i));
        // We retrieve the first position of this character before i:
        int j = 0;
        while (j < positions.size() && positions.get(j) >= i) {
            j++;
        }
        // If such a position exists (i.e. we have not reached the end of the list)
        // we set the shift to i-j.
        if (j < positions.size()) {
            //System.out.println(positions.get(j));
            bcrule = i-positions.get(j);
        }
        // Otherwise, the character does not appear anywhere
        // in the substring left of position i.
        // We can shift the pattern past the character at hand.
        else {
            bcrule = i;
        }
\end{minted}

Then, we apply the algorithm described in
Section~\ref{sec:good-suffix-rule} to get the shift for the good
suffix rule:

\begin{minted}{java}
        // Good suffix rule
        // Case 1: The suffix beginning at i has a copy,
        // which begins at f[i].
        if (f[i] >= 0) {
            gsrule = i - f[i];
        }
        // Case 2: The suffix has no copy in the pattern:
        // we look for the longest of its suffixes
        // which is also a prefix.
        else if (l[i] >= 0) {
            gsrule = l[i];
        }
        // Case 3: everything else has failed, we shift the pattern
        // by m places.
        else {
            gsrule = m;
        }
\end{minted}

And we return the maximum of the two computed shifts.

\subsection{Boyer-Moore main part}
\label{sec:boyer-moore-main}

After the preprocessing stage, the rest of the algorithm is completely
straightforward: we compare $q$ to the current substring of $S$ from
right to left, and we shift by the good amount when we encounter a
mismatch or the beginning of $q$.

\begin{minted}{java}
        int k = m-1;
        // Until we reach the end of the text:
        while (k < n) {
            int i = m-1;
            int h = k;
            // We compare the text to the pattern, from right to left.
            while (i >= 0 && s.charAt(h) == q.charAt(i)) {
                h--; i--;
            }
            // If we have reached the beginning of the pattern, we have found a match.
            if (i == -1) {
                res.add(k-m+1);
                // Shift by the biggest suffix which is also a prefix:
                if (l[1] >= 0) {
                    k = k + l[1];
                }
                // Or shift completely the pattern.
                else {
                    k = k + m + 1;
                }
            }
            else {
                // If the first character was different, we shift only by one place.
                if (i+1 == m) {
                    k++;
                }
                // Otherwise, we use the shift method.
                else {
                    k = k + shift(q, r, f, l, i+1);
                }
            }
        }
\end{minted}

\section{Test and performance analysis}
\label{sec:test-perf-analys-1}

Theoretically, and though we don't prove it here, Boyer-Moore runs in
linear time $\mathcal{O}(m+n)$. In worst-case complexity, it is equivalent to
Knuth-Morris-Pratt.

By testing on various examples the performances of Boyer-Moore against
Knuth-Morris-Pratt, we do not observe a significant difference when we
get to very long patterns.

For instance, for the same 1574-character pattern on the same
\emph{aspergillus fumigatus} genome, on the same computer, we get a
ratio of $1/4$ in computing time:

\begin{minted}{text}
********** Knuth-Morris-Pratt **********
Result: [29218957]
Time: 213 ms
****************************************

********** Boyer-Moore **********
Result: [29218957]
Time: 53 ms
********************************
\end{minted}

However, when the size of the alphabet grows, the Boyer-Moore
algorithm gets better. That is because it is based on character and
suffix repetitions. In a short alphabet, characters and substrings
occurs very frequently in the pattern, thus reducing the shifts made
by Boyer-Moore.

This algorithm outperforms Knuth-Morris-Pratt mainly when the alphabet
is huge (so the characters don't occur frequently) and when there are
few repetitions of the same substrings. Typically, Boyer-Moore is
ill-suited when it comes to DNA-matching.


\chapter{Conclusion}
\label{cha:conclusion}

As we have seen, none of the algorithms we implemented is universally
better than the others. Although the Boyer-Moore algorithm is the most
advanced and polyvalent of the three, both Karp-Rabin and
Knuth-Morris-Pratt have their uses. In the context of short patterns
with rare occurences, Karp-Rabin is extremely fast independently from
the size of the alphabet. The Knuth-Morris-Pratt algorithm is an
excellent compromise between simplicity and performance. While
Boyer-Moore is quite long to write and requires subtle thinking,
Knuth-Morris-Pratt is straightforward and can requires relatively few
lines of code; and matches the performance of Boyer-Moore in many
cases.

We have seen that Knuth-Morris-Pratt is better for short alphabets,
while Boyer-Moore may be more efficient with longer alphabets (e.g.\
for English text).

As such, based on the properties of the alphabet, the text and the
query, all three algorithms are worth considering when searching for a
pattern in a text. It all boils down to a question of compromises
rather than of one algorithm being intrisincally better than all of
the others.

Another category of algorithms worth mentioning is Ukkonen and Weiner,
both based on a new data structure called \emph{suffix trees}. Suffix
trees allow to look for a pattern in a text in $\mathcal{O}(m+n)$
time, like Knuth-Morris-Pratt and Boyer-Moore, but this time the
preprocessing (i.e.\ the building of the tree) is linear in the length
of the \emph{text}, while the search is linear in the length of the
\emph{pattern}. This allows for very fast searching of a multitude of
patterns in a single text. Ukkonen's and Weiner's algorithm allows one
to build such suffix trees in linear time.


%% Bibliographie

\nocite{*}
\bibliographystyle{alpha}
\bibliography{report}
\label{cha:bibliography}


\end{document}



%%% Local Variables:
%%% mode: latex
%%% TeX-master: t
%%% End:
